\documentclass[12pt]{report}
\usepackage[utf8]{inputenc}
\usepackage[english]{babel}
\usepackage{amsmath}
\usepackage{amsfonts}
\usepackage{titlesec}
\usepackage{todonotes}
\usepackage{float}
\usepackage{color}
\usepackage{listings}
\usepackage{lmodern}
\usepackage{graphicx}
\usepackage{bytefield}
\usepackage{tikz}
\usepackage[justification=centering,labelfont={small,bf},font={small,bf}]{caption}
\usepackage[scaled]{beramono} % BitStream Vera font (code)
\usepackage[T1]{fontenc}
\usepackage[a4paper, margin=2.54cm]{geometry}
\usepackage[nottoc,numbib]{tocbibind}

%\renewcommand{\familydefault}{\sfdefault}

\graphicspath{{figures/}}

\usepackage[
style=authoryear,
citestyle=authoryear,
backend=biber,
firstinits=true
]{biblatex}
\DeclareNameAlias{author}{last-first}

\addbibresource{../library.bib}
\DefineBibliographyStrings{english}{%
  references = {References},
}

\lstset{basicstyle=\footnotesize\ttfamily,breaklines=true}

\author{Amy Parent (1303985)}
\title{AHABus \& FCORE:\\Modular Platform for High Altitude Balloon Missions}
\date{April 2017}


% Parameters


\setlength{\baselineskip}{1.5em}
%\linespread{1.3em}

% Title


\makeatletter
\def\maketitle{%
    \vspace*{10em}
	\begin{center}%
	\let \footnote \thanks
        \vskip 5em%
		{\LARGE \textbf{\@title} \par}%
		\vskip 1.5em%
		{\large \@author}%
		\vskip .5em%
		{\normalsize School of Arts, Media and Games\\}%
		\vskip .1em%
		{\normalsize University of Abertay Dundee\\}%
		\vskip .1em%
		{\normalsize Dundee, DD1 1HG, UK}%
	\end{center}
}
\makeatother


\begin{document}

\begin{titlepage}
\maketitle
\end{titlepage}

\begin{abstract}
    
\end{abstract}

\newpage
\tableofcontents

\newpage
\chapter{Introduction}
\label{ch:introduction}

\newpage
\chapter{Literature Review}
\label{ch:literature-review}

Most HAB missions led by universities, schools or amateur are built around
commercially available small computers or micro-controller boards: Redland
Green School students used a Raspberry Pi (\cite{rpi2014}) running
python scripts as an On Board Computer (\textit{OBC}) for their HAB mission
(\cite{Hinschelwood2015}), while other projects have been ran on AVR
micro-controller based boards like the Arduino Mega
(\cite{AtmelCorporation2015}).

In the United Kingdom, radio licenses do not cover airborne transmission, making
packet radio protocols like AX.25 unusable for HAB missions 
(\cite{ukhasradio2016}). However, the Office of Communications allows the use
of certain \textit{license-exempt} frequencies like the 70cm wavelength band,
provided that the transmission power does not exceed 10mW (\cite{Ofcom2014}).
For this reason, most missions launched from the United Kingdom use the
Radiometrix NTX2 transmitter (\cite{radiometrix2012}). Because of the
constraints, these missions usually transmit their telemetry as text using the
Radio-Teletype (RTTY) protocol.

While most HAB projects make use of commercial-off-the-shelf components, the
finished payloads are usually custom, single-purpose designs that are little,
if at all modular. In the design document for the Titan-1 HAB mission, Bombasaro
describes a hardware bus based on tightly coupled sensors which communicate
over two different data buses, SPI and I2C, and a custom-made, single-purpose
flight software (\cite{Bombasaro2015}).

Because of the risks and cost involved, nano-satellites rely more on modular
designs, allowing each team to work on their module independently. In his paper,
Volstad describes the design of the data bus of the NTNU CubeSat: while the OBC
is a based on a custom made circuit board, it is designed to provide a standard
power interface, as well as access to an I2C standard bus
(\cite{NXPSemiconductors2014}) to each payload (\cite{Volstad2011}). The I2C
protocol was chosen because of its low power consumption, and because if only
requires two lines (clock and data). Thomas Clausen describes in his 2001 paper
how a simple packet protocol built on top of I2C itself is used for data
transfer and error detection in Aalborg University's CubeSats
(\cite{Clausen2001}).

Satellite missions last a lot longer than High-Altitude Balloons' and the
requirements for radio transmissions differ: while satellites can only
communicate when their orbit passes above a ground station and require precise
speed and data volume planning, as described by Sandy Anthunes in his book
(\cite{Antunes2015}), non-floating HAB missions allow for line-of-sight
communications from liftoff to late into the descent of the payload.

There are however some technologies that can be adapted from satellites to be
used in HAB applications. The University of Arizona uses a custom packet format
over 434MHz radio, containing raw binary values for each instrument's
measurements (\cite{Eatchel2002}). The BRITE-Austria CubeSat mission uses the
AX.25 packet radio protocol (used by amateur radio users), which allows the
use of off-the-shelf transmission and reception hardware rather than custom-made
circuits (\cite{Traussnig2007}).

Some nano-satellites use custom telemetry data format, which allows the team
to minimise the volume of data to be sent over radio. The Planetary Society's
Lightsail mission has an uplink connection that allows the ground station
to request specific logs or data. By default, The spacecraft only communicates
a custom beacon containing a summary of the its housekeeping data
(\cite{planetary2016}). Uplink being impractical with the transmission power
limits imposed in the United Kingdom, such a selective telemetry system cannot
be relied upon for HAB missions.

To improve collaboration and allow the deployment of large networks of
satellites, probes and other space vehicles, the Consultative Committee for
Space Data Systems has defined multiple communication standards that for
the different types of networks encountered in spacecrafts.

SpaceWire defines how instruments' data can be accessed by having the OBC
remotely poll their memory (\cite{parkes2005}). A similar system is used to
fetch data from ROM chips over I2C and could be used to build a simple data bus.

The CCSDS Space Packet protocol defines a packet format used
to encapsulate application data (from instruments for example) that can be sent
over a ``Space Link'', an analogue of the OSI model's link layer
(\cite{Stallings1987}) using CCSDS frames to encapsulate packets originating
from multiple instruments, ground stations and spacecrafts (\cite{ccsds2003}).

While HAB applications do not require the level of complexity of CCSDS
protocols — addressing is not needed since the network only contains two
endpoints, communicating in a single direction — some details of the Space
Packet protocol are worth reusing (Forward Error Correction, different
Application IDs for each payload).

\section{Analysis of the Problem}



\newpage
\chapter{Proposed Solution and Methodology}
\label{ch:methodology}

The proposed solution for this project is inspired by the bus model that
supports the architecture of most commercial satellites. A standard software
and hardware platform provides common facilities required on all High-Altitude
flights – position tracking, internal data transfers and radio telemetry to the
ground station – allowing payloads (scientific instruments for example) to
only generate data and make it available to the platform.

\begin{figure}[H]
\includegraphics[width=0.35\textwidth]{payloadbus.pdf}
\centering
\caption{Payload bus architecture}
\end{figure}

The platform, or \textit{Payload Bus} is composed of the following components hardware and software components:

\begin{description}
    
\item[On Board Computer:] the computer that runs the flight software and
controls the data bus and all communications units (also called
\textit{Bus Controller}).

\item[Flight Software:] the firmware installed on the \textit{OBC}, that
controls the bus and the execution of the mission.

\item[Power Bus:] the hardware interface that provides electrical power to
payloads.

\item[Data Bus:] the hardware interface and software protocol through which
payloads communicate data to the flight computer.

\item[Navigation System:] the hardware (GNSS receiver) and software module used
by the flight software to keep track of the balloon's geographic coordinates and
altitude.

\item[Radio System:] the hardware (transmitter), protocols and software module
used by the \textit{OBC} to transmit collected data and metadata to the
ground station.

\item[Ground Segment:] the software required to record and decode the signal
received from the platform.
    
\end{description}

The following chapter describes the design processes used to develop and
integrate those components into a functioning prototype of the Abertay
High-Altitude Bus (\textit{AHABus}), and the methods used to test the viability
and performance of the design.

\section{Radio Communications Protocol}

This section describes the design process that led to the radio communications
protocol used by the AHABus platform to send payload data and flight metadata
to the ground station. Its role is to ensure that arbitrary binary data – each
payload can use their own data format – as stably and reliably as possible. As
discussed in the following subsections, the design and implementation choice
were also driven by legal constraints: in the United Kingdom, airborne radio
transmitters are limited in frequency band (70cm wavelength) and power
(transmission power inferior to 10mW).

\subsection{Requirements}
\label{ssec:requirements}

Designing and building a communication layer that can be relied up was an
important requirement for the rest of the project. Transmitting as much as
possible of a mission's scientific data is primordial in case the platform
is carried too far by winds and cannot be recovered after landing.

From the constraints discussed above, the requirements below were established
for the radio communications system:

\begin{enumerate}
\item provide a system to transmit binary data.
\item minimise the amount of overhead and bandwidth required.
\item allow multiple payload's data to be sent on the same link.
\item provide stable transmissions on a low-power link.
\item correct as many transmission-induced data errors as possible.
\item allow ground crew to track the balloon's location over time.
\end{enumerate}

%TODO: mention one-way

As stated in chapter \ref{ch:literature-review}, the United Kingdom's Office
of Communications restrict what frequencies can be used for airborne
radio transmissions, and how much power the transmissions can be sent at.
Those restrictions constrain what existing protocols can be used to transmit
arbitrary data over radio. This disqualifies existing packet radio protocols
that would have fulfilled most of the requirements, in use in other countries
like AX.25.

Most UK-launched High Altitude Balloon missions make use of Radio-Teletype
(\textit{RTTY}) to send ASCII-encoded text. This was considered insufficient
for the purpose of a common platform for two main reasons: First, ASCII does
not handle binary data by default -- only some byte values would be displayed
as ASCII, while the rest would translate to non-printing characters. This leads
to data loss, unless the binary data is encoded in an ASCII-compatible format
like base64, which would require more bandwidth to send a similar amount of
data. Furthermore, ASCII alone does not provide either a way to multiplex data
coming from different payloads, or any kind of error checking or error
correction. This means that such facilities must be built on top of the
plain-text protocol.

RTTY itself functions in the same way as wired UART serial links. It can provide
some level of error checking if the transmission of a parity bit is enabled, but
it does not carry enough information to correct errors. Additionally, RTTY is
purely a link layer, and does not provide any kind of encapsulation necessary
for multiplexing data from different payloads.

\subsection{Protocol Specification}

The AHABus radio protocol was designed to match the requirements for the radio
system as closely as possible. While the specification presented in this section
is the last version, which was implemented in the AHABus prototype, some parts
of the protocol evolved or had to be changed during the prototyping process.
% TODO: explain more? less?

The protocol was designed in layers, some equivalent to the ones present in the
OSI model (\cite{Stallings1987}). This approach was chosen because it allows
some level of modularity – if a layer proves problematic, it can be swapped
for another with minimal impact on the other layers. Each layer fulfils a
specific purpose:

\begin{description}
\item[Link Layer:] The lowest-level layer's purpose is to establish a one-way
binary stream between the two end points of the radio link.

\item[Frame Layer:] The frame layer relies on the Link Layer to transmit its
data units, and provides reliability and error correction to higher-level 
layers.

\item[Packet Layer:] The packet layer is the highest-level layer. It provides
a way for multiple payloads' data to share bandwidth as well as metadata
transmission.
\end{description}

\subsubsection{Link Layer}

While RTTY was not considered sufficient to cover fulfil the requirements of
the whole protocol, as shown in section \ref{ssec:requirements}, it fulfils the
simple requirements of the Link Layer: it can be used to send 8-bit bytes, one
by one, over a radio link.

RTTY functions like UART, over radio. Instead of using logical electrical levels
(5V for logical high, 0V for logical low), it uses Audio Frequency Shift Keying
(\textit{AFSK}): two audio tones separated by a frequency shift act as two
logical levels. When nothing is being transmitted, the line is held high. The
start of a byte is signalled by the \textit{start bit} (the line goes low),
followed by the eight \textit{data bits}, and one or two \textit{stop bits} (the
line is held low). After a byte has been sent, the line is held high again until
the next byte, or the next byte starts immediately.

%TODO: replace mentions of UART with RS-232 // NOPE, RS232 diff. than TTL
%TODO: references for DominoEx and Olivia
Other existing protocols exist that can transmit data over radio audio signals,
like DominoEx or Olivia, sometimes with better transmission rates. RTTY was
preferred over them because the ESP8266 board used as OBC in the AHABus
prototype provides hardware support for UART, making it easier to implement and
reliable. Since the link layer is transparent to the frame and packet layers,
RTTY could be swapped for another protocol in the future with no impact on
those.

\subsubsection{Frame Layer}

\begin{figure}[H]
    \begin{bytefield}{16}
        \bitheader{0,7,8,15} \\
        \begin{rightwordgroup}{Header}
            \bitbox{8}{01011010} & \bitbox{8}{version} \\
            \bitbox{16}{sequence number}
        \end{rightwordgroup} \\
        \begin{leftwordgroup}{Payload}
            \wordbox[lrt]{1}{data} \\
            \skippedwords \\
            \wordbox[lrb]{1}{}
        \end{leftwordgroup} \\
        \begin{rightwordgroup}{FEC}
            \wordbox[lrt]{1}{checksum} \\
            \skippedwords \\
            \wordbox[lrb]{1}{}
        \end{rightwordgroup}
    \end{bytefield}
    \centering
    \caption{AHABus Frame Format}
\end{figure}

\subsubsection{Packet Layer}

\begin{figure}[H]
    \begin{bytefield}{32}
        \bitheader{0,7,8,15,16,23,24,31} \\
        \begin{rightwordgroup}{Header}
            \bitbox{8}{version} & \bitbox{8}{payload ID} &
            \bitbox{16}{packet length} \\
            \bitbox{32}{latitude} \\
            \bitbox{32}{longitude} \\
            \bitbox{16}{altitude} & \bitbox[lrt]{16}{}
        \end{rightwordgroup} \\
        \begin{leftwordgroup}{Payload}
            \wordbox[lrt]{1}{data} \\
            \skippedwords \\
            \wordbox[lrb]{1}{}
        \end{leftwordgroup}
    \end{bytefield}
    \centering
    \caption{AHABus Frame Format}
\end{figure}

\subsection{Initial Implementation and Testing}

\section{Bus and Flight Software}

\subsection{Requirements}

\subsection{Protocol Specification}

\subsection{Arduino Implementation}

\section{Flight Software}

\section{Ground Segment}

\subsection{Reception Binary Stream Extraction}

\section{Testing}

\subsection{Component Testing}

\subsection{Integrated Testing}

\subsubsection{Day in the life Testing}

\subsubsection{Resilience Testing}

\newpage
\chapter{Results}
\label{ch:reults}

\newpage
\chapter{Discussion}
\label{ch:discussion}

\newpage
\chapter{Conclusion}
\label{ch:conclusion}


\newpage
\printbibliography[heading=bibnumbered]  

\end{document}