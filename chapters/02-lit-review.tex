\chapter{Literature Review}
\label{ch:literature-review}

\section{Introduction}

\section{\acrlong{hab} Missions}

Most \acrshort{hab} missions led by universities, schools or amateur are built 
around commercially available small computers or micro-controller boards:
Redland Green School students used a Raspberry Pi (\cite{rpi2014}) running
python scripts as an \acrfull{obc} for their HAB mission
(\cite{Hinschelwood2015}), while other projects have been ran on AVR
micro-controller based boards like the Arduino Mega
(\cite{AtmelCorporation2015}).

\section{\acrlong{obc} Architectures}

While most \acrshort{hab} projects make use of commercial-off-the-shelf
components, the finished payloads are usually custom, single-purpose designs
that are little, if at all modular. In the design document for the Titan-1 HAB
mission, Bombasaro describes a hardware bus based on tightly coupled sensors
which communicate over two different data buses, SPI and \acrshort{i2c}, and a
custom-made, single-purpose flight software (\cite{Bombasaro2015}).

Because of the risks and cost involved, nano-satellites rely more on modular
designs, allowing each team to work on their module independently. In his paper,
Volstad describes the design of the data bus of the NTNU CubeSat: while the OBC
is a based on a custom made circuit board, it is designed to provide a standard
power interface, as well as access to an \acrshort{i2c} standard bus
(\cite{NXPSemiconductors2014}) to each payload (\cite{Volstad2011}). The
\acrshort{i2c} protocol was chosen because of its low power consumption, and
because if only requires two lines (clock and data). Thomas Clausen describes
in his 2001 paper how a simple packet protocol built on top of \acrshort{i2c}
itself is used for data transfer and error detection in Aalborg University's
CubeSats (\cite{Clausen2001}).

\section{Digital Radio Telemetry}

In the United Kingdom, radio licenses do not cover airborne transmission, making
packet radio protocols like AX.25 unusable for HAB missions 
(\cite{ukhasradio2016}). However, the Office of Communications allows the use
of certain \textit{license-exempt} frequencies like the 70cm wavelength band,
provided that the transmission power does not exceed 10mW (\cite{Ofcom2014}).
For this reason, most missions launched from the United Kingdom use the
Radiometrix NTX2 transmitter (\cite{radiometrix2012}). Because of the
constraints, these missions usually transmit their telemetry as text using the
\acrfull{rtty} protocol.

Satellite missions last a lot longer than High-Altitude Balloons' and the
requirements for radio transmissions differ: while satellites can only
communicate when their orbit passes above a ground station and require precise
speed and data volume planning, as described by Sandy Anthunes in his book
(\cite{Antunes2015}), non-floating HAB missions allow for line-of-sight
communications from liftoff to late into the descent of the payload.

There are however some technologies that can be adapted from satellites to be
used in \acrshort{hab} applications. The University of Arizona uses a custom
packet format over 434MHz radio, containing raw binary values for each
instrument's measurements (\cite{Eatchel2002}). The BRITE-Austria CubeSat
mission uses the AX.25 packet radio protocol (used by amateur radio users),
which allows the use of off-the-shelf transmission and reception hardware
rather than custom-made circuits (\cite{Traussnig2007}).

\section{Telemetry Protocols}

Some nano-satellites use custom telemetry data format, which allows the team
to minimise the volume of data to be sent over radio. The Planetary Society's
Lightsail mission has an uplink connection that allows the ground station
to request specific logs or data. By default, The spacecraft only communicates
a custom beacon containing a summary of the its housekeeping data
(\cite{planetary2016}). Uplink being impractical with the transmission power
limits imposed in the United Kingdom, such a selective telemetry system cannot
be relied upon for \acrshort{hab} missions.

To improve collaboration and allow the deployment of large networks of
satellites, probes and other space vehicles, the Consultative Committee for
Space Data Systems has defined multiple communication standards that for
the different types of networks encountered in spacecrafts.

SpaceWire defines how instruments' data can be accessed by having the OBC
remotely poll their memory (\cite{parkes2005}). A similar system is used to
fetch data from ROM chips over \acrshort{i2c} and could be used to build a
simple data bus.

The \acrshort{ccsds} Space Packet protocol defines a packet format used to
encapsulate application data (from instruments for example) that can be sent
over a ``Space Link'', an analogue of the OSI model's link layer
(\cite{Stallings1987}) using \acrshort{ccsds} frames to encapsulate packets
originating from multiple instruments, ground stations and spacecrafts
(\cite{ccsds2003}).

While \acrshort{hab} applications do not require the level of complexity of
\acrshort{ccsds} protocols — addressing is not needed since the network only
contains two endpoints, communicating in a single direction — some details of
the Space Packet protocol are worth reusing (Forward Error Correction,
different Application IDs for each payload).

\section{Summary}