\chapter{Discussion}
\label{ch:discussion}

The aim of the project was to design and build a prototype of a High-Altitude
Balloon modular platform, similar in principle to buses used in the satellite
industry, and evaluate both the feasibility of such a platform and its
performance. The platform was required to allow for stable internal
communications between the payloads and the \acrlong{obc}, tracking of
ancillary data (\acrshort{gnss} co-ordinates and altitude) and reliable
transmission of the collected telemetry to one or multiple ground stations 
through a low-power radio channel. In this chapter, the \acrshort{ahabus}
platform prototype is evaluated against these aims, first qualitatively through
general observations and critical analysis, then in quantitatively, using the
testing results presented in chapter \ref{ch:results}. This chapter finally
presents the further work that would be required for the platform to fully meet
its design goals.

\section{Project Evaluation}

\subsection{Platform}

As a software platform, the \acrshort{ahabus} prototype fulfils most of its
goals: the flight \acrshort{fcore} was complete enough to support full-flight
simulations, manage the bus, \acrshort{gnss} and radio communications without
failure. During the testing phase, no major software design error were found,
and the only instances where failures were found to be caused by the flight
software itself. Though quite cumbersome to work with, and lacking in
documentation, the \acrshort{fcore} configuration tool discussed in section
\ref{sssec:conf-file} proved reliable, and the resulting \acrshort{fcore} builds
did not suffer from erroneous bandwidth usage.

There are, however, areas that could be improved upon: the software design of
\acrshort{fcore} makes use of numerous data buffers for the different
transmission layers. Those buffers are required because of the slow speeds that
those communications protocol use, but are at the moment not optimised to be as
small as required – instead, all buffers can hold \SI{4096}{\byte}. This would
prevent the current source code to be ported to architectures that provide less
dynamic memory than the chosen ESP8266 micro-controller does.

Due to lack of time, very little work was done on the hardware aspect of
\acrshort{ahabus}: testing was done using a development board instead of the
bare micro-controller for the \acrlong{obc}, and the power and payload bus were
little more than wires routed through a breadboard. An improved development
platform using a printed circuit board was designed as presented in
\ref{sec:flight-hardware}, but could not be fully assembled and tested in time.

\subsection{Radio Communications System}

\subsection{Payload Bus}

\subsection{Ground Segment}

\section{Analysis of Test Results}

\subsection{Day-in-the-Life Test}

\subsection{Radio System Resilience Tests}

\section{Further Work} % << mix in with the rest?