\chapter{Results}
\label{ch:results}

\section{Component Testing}

\section{Day In The Life Testing}

The Day-in-the-Life test consisted of a full simulated \acrlong{hab} mission,
from lift-off to landing. To obtain data similar to what would be encountered
during a real flight, the GNSS receiver was replaced by an Arduino board 
sending simulated geographic co-ordinates and altitude.

Figure \ref{fig:dit-alt} shows the altitude of the AHABus platform reported by
the received headers over the approximately two hours and twenty minutes of the
simulation. The altitude marks are matched to the time at which each packet
is received, and not emitted.

\begin{figure}[H]
\begin{tikzpicture}
\begin{axis}
\addplot table [
xlabel=Mission Elapsed Time,
ylabel=Altitude,
x=met, y=alt, col sep=comma
] {data/02-ditl/loc.csv};
\end{axis}
\end{tikzpicture}
\centering
\caption{DITL Test - Altitude over time}
\label{fig:dit-alt}
\end{figure}

The mission platform was composed of the AHABus \acrlong{obc}, and two simulated
payloads: payload 10 at priority level 2, and payload 20 at priority level 1.
Table \ref{tab:ditl-cfg}.

\begin{table}[H]
\caption {DITL Test - FCORE configuration}
\label{tab:ditl-cfg}
\begin{center}
\begin{tabular}{ |c|c c c| }
\hline
payload & priority & core frequency & polling interval \\
\hline
GNSS & - & -
10 & 2 & \SI{5.082}{\hertz} & \SI{102}{\second} \\
20 & 1 & \SI{10.164}{\hertz} & \SI{51}{\second} \\
\hline 
\end{tabular}
\end{center}
\end{table}



\section{Radio Stress-Testing}