\chapter{Results}
\label{ch:results}

\section{Component Testing}

\section{Day In The Life Testing}

The Day-in-the-Life test consisted of a full simulated \acrlong{hab} mission,
from lift-off to landing. To obtain data similar to what would be encountered
during a real flight, the GNSS receiver was replaced by an Arduino board 
sending simulated geographic co-ordinates and altitude.

Figure \ref{fig:dit-alt} shows the altitude of the AHABus platform reported by
the received headers over the approximately two hours and twenty minutes of the
simulation. The altitude marks are matched to the time at which each packet
is received, and not emitted.

\begin{figure}[H]
\begin{tikzpicture}
\begin{axis} [
xlabel=Mission Elapsed Time (\SI{}{\second}),
ylabel=Altitude (\SI{}{\metre}),
mark size=1.0pt,
mark repeat={5},
legend style ={
    at={(0.98,0.95)}, 
    anchor=north east, draw=black, 
    fill=white,align=left
}
]
\addplot+ table [
x=MET, y=altitude, col sep=comma
] {data/test-ditl/loc.csv};
\end{axis}
\end{tikzpicture}
\centering
\caption{DITL Test - Altitude over time}
\label{fig:dit-alt}
\end{figure}

% The mission platform was composed of the AHABus \acrlong{obc}, and two simulated
% payloads: payload 10 at priority level 2, and payload 20 at priority level 1.
% Table \ref{tab:ditl-cfg}.
%
% \begin{table}[H]
% \begin{center}
% \begin{tabular}{ |c|c c c| }
% \hline
% payload & priority & core frequency & polling interval \\
% \hline
% GNSS & - & - & - \\
% 10 & 2 & \SI{5.082}{\hertz} & \SI{102}{\second} \\
% 20 & 1 & \SI{10.164}{\hertz} & \SI{51}{\second} \\
% \hline
% \end{tabular}
% \end{center}
% \caption {DITL Test - FCORE configuration}
% \label{tab:ditl-cfg}
% \end{table}

Figure \ref{fig:ditl-bandwidth} shows that the average data rate used by the
platform stabilised early into the flight close to \SI{10}{bps}, or
\SI{110}{bps} (out of a maximum bandwidth of \SI{200}{bps}). Over the full
simulation's duration, an average of \SI{0.63}{Bps} were lost due to
transmission errors.

\begin{figure}[H]
\begin{tikzpicture}
\begin{axis} [
xlabel=Mission Elapsed Time (\SI{}{\second}),
ylabel=Data Rate (\SI{}{Bps}),
mark size=1.0pt,
mark repeat={5},
legend style ={
    at={(0.98,0.95)}, 
    anchor=north east, draw=black, 
    fill=white,align=left
}
]
\addplot+ table [
x=MET, y=error rate, col sep=comma
] {data/test-ditl/fcore.log.rxstats.csv};

\addplot+ table [
x=MET, y=bandwidth, col sep=comma
] {data/test-ditl/fcore.log.rxstats.csv};

\legend{Erroneous, Total}
\end{axis}
\end{tikzpicture}
\centering
\caption{DITL Test - Average data rates}
\label{fig:ditl-bandwidth}
\end{figure}

%
%
% \begin{figure}[H]
% \begin{tikzpicture}
% \begin{axis} [
% xlabel=Mission Elapsed Time (\SI{}{\second}),
% ylabel=Received Data (\SI{}{B}),
% mark size=1.0pt,
% mark repeat={5},
% legend style ={
%     at={(0.02,0.95)},
%     anchor=north west, draw=black,
%     fill=white,align=left
% }
% ]
% \addplot+ table [
% x=MET, y=received, col sep=comma
% ] {data/test-ditl/fcore.log.rxstats.csv};
%
% \addplot+ table [
% x=MET, y=valid, col sep=comma
% ] {data/test-ditl/fcore.log.rxstats.csv};
%
% \addplot+ table [
% x=MET, y=invalid, col sep=comma
% ] {data/test-ditl/fcore.log.rxstats.csv};
%
% \legend{Total, Valid Frame Data, Invalid Frame Data}
% \end{axis}
% \end{tikzpicture}
% \centering
% \caption{DITL Test - Received data}
% \label{fig:ditl-rxstats}
% \end{figure}

For the simulated mission, FCORE was configured to transmit a health packet
every \SI{120}{\second}, reporting the status of the GNSS receiver and the
two payloads. Table \ref{tab:ditl-health} presents the data extracted from those
packets: both payloads were available and responding to the bus controller more
than \SI{90}{\percent} of the time. The actual percentage could differ by a few
percentage points since the system reports health at fairly large time
intervals. It is interesting to note that the GNSS receiver is marked as
recovering in nearly \SI{3}{\percent} of the health messages, indicating an
inability from the \acrshort{ahabus} platform to parse a valid location from
its input.

\begin{table}[H]
\begin{center}
\begin{tabular}{ |c|c c c c| }
\hline
payload & up count & down count & recovering count & uprate \\
\hline
GNSS       & 67 & 0 & 2 & 97.1\% \\
payload 10 & 64 & 0 & 5 & 92.8\% \\
payload 20 & 63 & 0 & 6 & 91.3\% \\
\hline 
\end{tabular}
\end{center}
\caption {DITL Test - Platform health data}
\label{tab:ditl-health}
\end{table}

\section{Radio Stress-Testing}

The final series of tests focused on the radio communications system's
performance and ability to correct noise-induced errors in the received data.
As discussed in section \ref{sec:radio-coms}, the \acrfull{ecc} embedded in
each frame can correct up to sixteen erroneous bytes per frame of 256 bytes.
A modified version of FCORE was used to corrupt given amounts of bytes before
transmission and thus simulate a noise radio channel. The three tests were run
for ten minutes each with increasing error simulation rates below, near and
above the theoretical limit of the \acrshort{fec} algorithm, at $\frac{8}{256}$,
$\frac{14}{256}$ and $\frac{18}{256}$ error rates.

Figure \ref{fig:stress-8.256} shows that when the byte error rate is well below
(half) the theoretical maximum acceptable, the \acrshort{fec} algorithm performs
well and only a single frame was lost, due to an antenna disconnection.

\begin{figure}[H]
\begin{tikzpicture}
\begin{axis} [
xlabel=Mission Elapsed Time (\SI{}{\second}),
ylabel=Received Data (\SI{}{\byte}),
mark size=1.0pt,
legend style ={
    at={(0.02,0.95)}, 
    anchor=north west, draw=black, 
    fill=white,align=left
}
]
\addplot+ table [
x=MET, y=received, col sep=comma
] {data/test-rs8.256/fcore.log.rxstats.csv};

\addplot+ table [
x=MET, y=invalid, col sep=comma
] {data/test-rs8.256/fcore.log.rxstats.csv};

\addplot+ table [
x=MET, y=fixed, col sep=comma
] {data/test-rs8.256/fcore.log.rxstats.csv};

\legend{Total, Erroneous, Fixed}
\end{axis}
\end{tikzpicture}
\centering
\caption{$\frac{8}{256}$ Error Test - Received Data}
\label{fig:stress-8.256}
\end{figure}

When the average byte error rate, as shown in figure \ref{fig:stress-14.256},
some frames can be corrected, while some where the error rate is higher than
the average are lost. The fluctuations in actual error rate are due to the use
of the C language's \texttt{rand()} function. %remote or explain

\begin{figure}[H]
\begin{tikzpicture}
\begin{axis} [
xlabel=Mission Elapsed Time (\SI{}{\second}),
ylabel=Received Data (\SI{}{\byte}),
mark size=1.0pt,
legend style ={
    at={(0.02,0.95)}, 
    anchor=north west, draw=black, 
    fill=white,align=left
}
]
\addplot+ table [
x=MET, y=received, col sep=comma
] {data/test-rs14.256/fcore.log.rxstats.csv};

\addplot+ table [
x=MET, y=invalid, col sep=comma
] {data/test-rs14.256/fcore.log.rxstats.csv};

\addplot+ table [
x=MET, y=fixed, col sep=comma
] {data/test-rs14.256/fcore.log.rxstats.csv};

\legend{Total, Erroneous, Fixed}
\end{axis}
\end{tikzpicture}
\centering
\caption{$\frac{14}{256}$ Error Test - Received Data}
\label{fig:stress-14.256}
\end{figure}

Finally, an average error rate above the theoretical maximum of \SI{16}{\byte}
causes most of the frames to be lost as shown in figure \ref{fig:stress-18.256}.

\begin{figure}[H]
\begin{tikzpicture}
\begin{axis} [
xlabel=Mission Elapsed Time (\SI{}{\second}),
ylabel=Received Data (\SI{}{\byte}),
mark size=1.0pt,
legend style ={
    at={(0.02,0.95)}, 
    anchor=north west, draw=black, 
    fill=white,align=left
}
]
\addplot+ table [
x=MET, y=received, col sep=comma
] {data/test-rs18.256/fcore.log.rxstats.csv};

\addplot+ table [
x=MET, y=invalid, col sep=comma
] {data/test-rs18.256/fcore.log.rxstats.csv};

\addplot+ table [
x=MET, y=fixed, col sep=comma
] {data/test-rs18.256/fcore.log.rxstats.csv};

\legend{Total, Erroneous, Fixed}
\end{axis}
\end{tikzpicture}
\centering
\caption{$\frac{18}{256}$ Error Test - Received Data}
\label{fig:stress-18.256}
\end{figure}

