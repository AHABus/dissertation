\chapter{Results}
\label{ch:results}

\section{Introduction}

The following section presents the observations and numerical results obtained
during the tests done as part of the evaluation process of the project, as
discussed in section \ref{sec:testing}.

\section{General Observations}

Overall, the \acrshort{ahabus} prototype met expectations and performed well.
While some minor issues were encountered during the initial round of testing,
both endpoints of the platform – flight platform and ground segment – functioned
well enough that the next round of testing (range and flight tests) would be
achievable.

\subsection{Radio Communications}
\label{ssec:results-rtx}

The initial tests of the radio encoder and transmission stack ported to the
ESP8266 revealed issues with the micro-controller's interrupt system: events for
both \acrshort{uart} share the same interrupt, and \texttt{UART0} events (used
when communicating with the \acrshort{gnss} receiver) were detected at the same
time as \texttt{UART1} events. Additionally, the design of the ESP8266 meant
that the interrupts would fire not just when the transmission buffer crossed
its low threshold, but whenever it was filled below the threshold, locking the
controller into an interrupt loop and eventually triggering the watchdog timer.

Once those issues were fixed, as discussed in section \ref{sec:flight-software},
the radio protocol performed without issues on the transmission side. Reception
proved mostly stable apart from one main issue caused by the frame format: since
the end of a frame is detected by the number of byte received, byte losses
while receiving a frame would cause the start of the subsequent frame to be
counted as the end of the current one. As a result, dropped bytes in a frame
would most often cause the next frame to not be detected even if it was formed
properly.

\subsection{Payload Bus}
\label{ssec:results-bus}

Though the payload bus was found to be reliable overall, two issues were found
during platform testing. First, the default \acrshort{i2c} library provided
with the ESP8266 SDK had a very short clock stretch timeout, which meant that
payloads had a very short time window to respond to any request from the bus
controller, leading to a high rate of failed communications until the stretch
value was raised. Second, there were some cases where the last command from the
bus controller (clear the connection flag register) would fail. As a result,
the payload would be prevented to write any new data until the next
communication from the bus controller.

\section{Day In The Life Testing}
\label{ssec:results-ditl}

The Day-in-the-Life test consisted of a full simulated \acrlong{hab} mission,
from lift-off to landing. To obtain data similar to what would be encountered
during a real flight, the GNSS receiver was replaced by an Arduino board 
sending simulated geographic co-ordinates and altitude.

Figure \ref{fig:ditl-alt} shows the altitude of the AHABus platform reported by
the received headers over the approximately two hours and twenty minutes of the
simulation. The altitude marks are matched to the time at which each packet
is received, and not emitted.

\begin{figure}[h]
\begin{tikzpicture}
\begin{axis} [
xlabel=Mission Elapsed Time (\SI{}{\second}),
ylabel=Altitude (\SI{}{\metre}),
mark size=1.0pt,
mark repeat={5},
legend style ={
    at={(0.98,0.95)}, 
    anchor=north east, draw=black, 
    fill=white,align=left
}
]
\addplot+ table [
x=MET, y=altitude, col sep=comma
] {data/test-ditl/loc.csv};
\end{axis}
\end{tikzpicture}
\centering
\caption{DITL Test - Altitude over time}
\label{fig:ditl-alt}
\end{figure}

The mission platform was composed of the AHABus \acrlong{obc}, and two simulated
payloads: payload 10 at priority level 2, and payload 20 at priority level 1.
Table \ref{tab:ditl-cfg} shows the core frequency and polling intervals for
each payload, generated from the mission file by the FCORE configuration tool.

\begin{table}[h]
\begin{center}
\begin{tabular}{c||c c c}

payload & priority & core frequency & polling interval \\ \hline
10 & 2 & \SI{5.082}{\hertz} & \SI{102}{\second} \\
20 & 1 & \SI{10.164}{\hertz} & \SI{51}{\second} \\

\end{tabular}
\end{center}
\caption {DITL Test - FCORE configuration}
\label{tab:ditl-cfg}
\end{table}

Figure \ref{fig:ditl-bandwidth} shows that the average data reception rate was
stable over the whole simulation, with an average of \SI{11}{Bps}, or
\SI{121}{bps} (out of a maximum bandwidth of \SI{200}{bps}). Over the full
simulation's duration, an average of \SI{0.63}{Bps} were lost due to
transmission errors.

\begin{figure}[h]
\begin{tikzpicture}
\begin{axis} [
xlabel=Mission Elapsed Time (\SI{}{\second}),
ylabel=Received Data (\SI{}{\byte}),
mark size=1.0pt,
mark repeat={10},
legend style ={
    at={(0.02,0.95)}, 
    anchor=north west, draw=black, 
    fill=white,align=left
}
]

\addplot+ table [
x=MET, y=received, col sep=comma
] {data/test-ditl/fcore.log.rxstats.csv};

\addplot+ table [
x=MET, y=invalid, col sep=comma
] {data/test-ditl/fcore.log.rxstats.csv};

\legend{Total, Erroneous}
\end{axis}
\end{tikzpicture}
\centering
\caption{DITL Test - Received frame data}
\label{fig:ditl-bandwidth}
\end{figure}

For the simulated mission, FCORE was configured to transmit a health packet
every \SI{120}{\second}, reporting the status of the GNSS receiver and the
two payloads. Table \ref{tab:ditl-health} presents the data extracted from those
packets: both payloads were available and responding to the bus controller more
than \SI{90}{\percent} of the time. The actual percentage could differ by a few
percentage points since the system reports health at fairly large time
intervals. It is interesting to note that the GNSS receiver is marked as
recovering in nearly \SI{3}{\percent} of the health messages, indicating an
inability from the \acrshort{ahabus} platform to parse a valid location from
its input.

\begin{table}[h]
\begin{center}
\begin{tabular}{c||c c c|c}

payload & up count & down count & recovering count & uptime ratio \\ \hline
GNSS       & 67 & 0 & 2 & 97.1\% \\
payload 10 & 64 & 0 & 5 & \SI{92.8}{\percent} \\
payload 20 & 63 & 0 & 6 & \SI{91.3}{\percent} \\

\end{tabular}
\end{center}
\caption {DITL Test - Platform health data}
\label{tab:ditl-health}
\end{table}

The number of valid and invalid packets received for each payload are given in
table \ref{tab:ditl-packet-count}, along with the expected number of packets
calculated from the FCORE configuration polling intervals given in table
\ref{tab:ditl-cfg} and the mission duration of \SI{8482}{\second}. The ratio of
received packets to expected packets matches the uptime ratios reported by
FCORE health packets, given in table \ref{tab:ditl-health}.

\begin{table}[h]
\begin{center}
\begin{tabular}{c||c c|c c|c}

payload & valid packets & invalid packets & total & expected & ratio \\
\hline
payload 10 & 71 & 2 & 79 & 83 & \SI{95.2}{\percent} \\
payload 20 & 138 & 6 & 144 & 166 & \SI{86.7}{\percent} \\

\end{tabular}
\end{center}
\caption {DITL Test - Received packets}
\label{tab:ditl-health}
\end{table}

\section{Radio System Resilience Testing}
\label{ssec:results-resilience}

The final series of tests focused on the radio communications system's
performance and ability to correct noise-induced errors in the received data.
As discussed in section \ref{sec:radio-coms}, the \acrfull{ecc} embedded in
each frame can correct up to sixteen erroneous bytes per frame of 256 bytes.
A modified version of FCORE was used to corrupt given amounts of bytes before
transmission and thus simulate a noise radio channel. The three tests were run
for ten minutes each with increasing error simulation rates below, near and
above the theoretical limit of the \acrshort{fec} algorithm, at $\frac{8}{256}$,
$\frac{14}{256}$ and $\frac{18}{256}$ error rates.

Figure \ref{fig:stress-8.256} shows that when the byte error rate is well below
(half) the theoretical maximum acceptable, the \acrshort{fec} algorithm performs
well and only a single frame was lost, due to an antenna disconnection.

\begin{figure}[h]
\begin{tikzpicture}
\begin{axis} [
xlabel=Mission Elapsed Time (\SI{}{\second}),
ylabel=Received Data (\SI{}{\byte}),
mark size=1.0pt,
legend style ={
    at={(0.02,0.95)}, 
    anchor=north west, draw=black, 
    fill=white,align=left
}
]
\addplot+ table [
x=MET, y=received, col sep=comma
] {data/test-rs8.256/fcore.log.rxstats.csv};

\addplot+ table [
x=MET, y=invalid, col sep=comma
] {data/test-rs8.256/fcore.log.rxstats.csv};

\legend{Total, Erroneous}
\end{axis}
\end{tikzpicture}
\centering
\caption{$\frac{8}{256}$ Error Test - Received Data}
\label{fig:stress-8.256}
\end{figure}

When the average byte error rate, as shown in figure \ref{fig:stress-14.256},
some frames can be corrected, while some where the error rate is higher than
the average are lost. The fluctuations in actual error rate are due to the use
of the C language's \texttt{rand()} function. %remote or explain

\begin{figure}[h]
\begin{tikzpicture}
\begin{axis} [
xlabel=Mission Elapsed Time (\SI{}{\second}),
ylabel=Received Data (\SI{}{\byte}),
mark size=1.0pt,
legend style ={
    at={(0.02,0.95)}, 
    anchor=north west, draw=black, 
    fill=white,align=left
}
]
\addplot+ table [
x=MET, y=received, col sep=comma
] {data/test-rs14.256/fcore.log.rxstats.csv};

\addplot+ table [
x=MET, y=invalid, col sep=comma
] {data/test-rs14.256/fcore.log.rxstats.csv};

\legend{Total, Erroneous}
\end{axis}
\end{tikzpicture}
\centering
\caption{$\frac{14}{256}$ Error Test - Received Data}
\label{fig:stress-14.256}
\end{figure}

Finally, an average error rate above the theoretical maximum of \SI{16}{\byte}
causes most of the frames to be lost as shown in figure \ref{fig:stress-18.256}.

\begin{figure}[h]
\begin{tikzpicture}
\begin{axis} [
xlabel=Mission Elapsed Time (\SI{}{\second}),
ylabel=Received Data (\SI{}{\byte}),
mark size=1.0pt,
legend style ={
    at={(0.02,0.95)}, 
    anchor=north west, draw=black, 
    fill=white,align=left
}
]
\addplot+ table [
x=MET, y=received, col sep=comma
] {data/test-rs18.256/fcore.log.rxstats.csv};

\addplot+ table [
x=MET, y=invalid, col sep=comma
] {data/test-rs18.256/fcore.log.rxstats.csv};

\legend{Total, Erroneous}
\end{axis}
\end{tikzpicture}
\centering
\caption{$\frac{18}{256}$ Error Test - Received Data}
\label{fig:stress-18.256}
\end{figure}

Figure \ref{fig:stress-fec} shows that in all three tests, the amount of bytes
that were corrected using the frames' \acrlong{ecc} was similar. As expected,
the test with the highest byte error rate shows a slightly inferior correction
rate – frames that contain more than 16 erroneous bytes cannot be corrected at
all.

\begin{figure}[h]
\begin{tikzpicture}
\begin{axis} [
xlabel=Mission Elapsed Time (\SI{}{\second}),
ylabel=Error-Corrected Data (\SI{}{\byte}),
mark size=1.0pt,
legend style ={
    at={(0.02,0.95)}, 
    anchor=north west, draw=black, 
    fill=white,align=left
}
]

\addplot+ table [
x=MET, y=fixed, col sep=comma
] {data/test-rs8.256/fcore.log.rxstats.csv};

\addplot+ table [
x=MET, y=fixed, col sep=comma
] {data/test-rs14.256/fcore.log.rxstats.csv};

\addplot+ table [
x=MET, y=fixed, col sep=comma
] {data/test-rs18.256/fcore.log.rxstats.csv};

\legend{$\frac{8}{256}$, $\frac{14}{256}$, $\frac{18}{256}$}
\end{axis}
\end{tikzpicture}
\centering
\caption{$\frac{14}{256}$ Error Tests - Error Correction Performance}
\label{fig:stress-fec}
\end{figure}

Table \ref{tab:stress-fec} shows that the reported error correction rates
are slightly less than the theoretical error rates induced at transmission as
long as the these theoretical rates are below the maximum acceptable rate, and
fall well below when the error rate is above the limit.

\begin{table}[h]
\begin{center}
\begin{tabular}{c||c c}

test & Correction Rate & Theoretical Error Rate \\ \hline
$\frac{8}{256}$  & \SI{2.826}{\percent} & \SI{3.125}{\percent} \\ [0.2em]
$\frac{14}{256}$ & \SI{3.655}{\percent} & \SI{5.469}{\percent} \\ [0.2em]
$\frac{18}{256}$ & \SI{2.514}{\percent} & \SI{7.031}{\percent} \\ [0.2em]

\end{tabular}
\end{center}
\caption {DITL Test - Platform health data}
\label{tab:stress-fec}
\end{table}

